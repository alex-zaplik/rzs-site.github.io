\documentclass{article}

\usepackage{polski}
\usepackage[utf8]{inputenc}
\usepackage[T1]{fontenc}

\usepackage[margin=1.5in]{geometry} 
\usepackage{color} 
\usepackage{amsmath}                                                                      
\usepackage{amsfonts}                                                                      
\usepackage{graphicx}                                                                     
\usepackage{booktabs}
\usepackage{amsthm}
\usepackage{pdfpages}
\usepackage{wrapfig}

\theoremstyle{definition}
\newtheorem{de}{Definicja}[subsection]

\theoremstyle{definition}
\newtheorem{tw}{Twierdzenie}[subsection]

\theoremstyle{definition}
\newtheorem*{fakt}{FAKT}

\author{Jakub Gogola, Mikołaj Pietrek, Paweł Wilkosz}
\title{Analiza Matematyczna 1 - powtórzenie}  



\begin{document}

\maketitle

\section{Wstęp}

Drogi czytelniku! Jeśli to czytasz, prawdopodobnie tak jak ja stoisz u wrót
egzaminu a Analizy Matematycznej 1 z prof. Pawłem Krupskim
i szukasz źródła które pomoże co w powtórzeniu
całej teorii potrzebnej żeby zdać ten piękny przedmiot. W tym celu postanowiłem
zrobić tego PDFa :---) (\#pdk hehe) jest to zapis notatek Jakuba Gogoli
z pewnymi zmianami poczynionymi przeze mnie. Dołączam także kartę wzorów autorstwa
Mikołaja Pietrka.
Zastrzegam iż jest to projekt non-profit
(z czego 50\% obiecałem Jakubowi Gogoli i Mikołajowi
Pietrkowi).

NIE GWARANTUJĘ SKUTECZNOŚCI PONIŻSZYCH NOTATEK, JEŻELI CHCESZ MIEĆ 100\% 
PEWNOŚĆ POPRAWNOŚCI MATERIAŁU, ZAJRZYJ DO LITERATURY!!!

\section{Ciągi}

\begin{de}
(Ciąg liczb rzeczywistych)

Ciągiem liczb rzeczywitych nazywamy dowolną funkcję
$ f: \mathbb{N} \rightarrow \mathbb{R} $
Wartości $ f(n) $ to wyrazy ciągu.
\end{de}

\begin{de}
(Granica ciągu)

Liczba $ a \in \mathbb{R} $ jest granicą ciągu $ (a_n) $, gdy:
$$
(\forall \epsilon > 0)(\exists k \in \mathbb{N})(\forall n \geq k)
(|a_n - a| < \epsilon)
$$

Mówimy wtedy, że ciąg jest zbieżny do granicy $a$, czyli
$$ \lim_{n \to \infty} a_n = a $$

Lub prościej: $ \lim a_n =a $
\end{de}

\subsection{Własności ciągów zbieżnych}

\begin{enumerate}
\item Każdy ciąg zbieżny ma dokładnie jedną granicę.
\item ciąg zbieżny jest ograniczony jeśli
      $ \lim a_n = a $ to
      $ (\exists k)(\forall n \geq k)(|a_n - a| < 1) $
\item $ \lim |a_n| = |\lim a_n| $
\end{enumerate}

\begin{tw}
(o trzech ciągach)

Jeżeli $ a_n \leq c_n \leq b_n $ dla dostatecznie dużych n i 
$ \lim a_n = \lim b_n $ to
$ \lim c_n = \lim a_n = \lim b_n $
\end{tw}

\begin{tw}
Każdy ciąg \textbf{monotoniczny} i \textbf{ograniczony}
jest \textbf{zbieżny}
\end{tw}

\begin{de}
(ciąg rozbieżny)

Ciąg jest rozbieżny do $ \infty $, gdy
$$
(\forall r \in \mathbb{R})(\exists k)(\forall n \geq k)(a_n > r)
$$

Ciąg jest rozbieżny do $ -\infty $, gdy
$$
(\forall r \in \mathbb{R})(\exists k)(\forall n \geq k)(a_n < r)
$$
\end{de}

\begin{tw}
\
\begin{enumerate}
\item Jeśli $ \lim a_n = \pm \infty $ to 
      $ \lim \frac{1}{a_n} = 0 $
\item Jeśli $(\lim a_n) = 0$, $a_n > 0$ dla prawie wszystkich n
      ( $ (\exists N) (\forall n>N) $ )
      to $ \lim \frac{1}{a_n} = + \infty $
      (analogicznie $-\infty$ dla $a_n < 0$)
\item Jeśli $(a_n)$ jest \textbf{ograniczony} i
      $ \lim (b_n = \infty) $
      to $ \lim (a_n + b_n) = \infty $\\
      oraz $ \lim (a_n - b_n) = - \infty $
      i $ \lim (\frac{a_n}{b_n}) = 0 $
\item Jeśli $ \lim a_n = a $, $a>0$, $ \lim b_n = \infty $,
      to $ \lim a_n b_n = \infty $
      (analogicznie gdy $\lim b_n = - \infty $)
\end{enumerate}
\end{tw}

\begin{de}
(podciąg)

Jeśli $ (\forall n \in \mathbb{N}) m_1<m_2<...<m_n \in \mathbb{N} $,
to $ (a_{m_n})_{n=1}^{\infty} $
nazywamy podciągiem ciągu $ (a_n)_{n=1}^{\infty} $
\end{de}

\begin{tw}
Podciąg ciągu zbieżnego do $a$ jest zbieżny do $a$.
\end{tw}

\begin{tw}
(Bolzano-Weierstrasa)

Każdy ograniczony ciąg liczb rzeczywistych ma podciąg zbieżny.

\underline{WNIOSEK}
\begin{enumerate}
\item Każdy ciąg ograniczony $ (a_n) $ taki, że każdy jego podciąg jest
      zbieżny do granicy g jest też zbieżny do granicy g.
\item Każdy ciąg ograniczony rozbieżny zawiera przynajmniej 
      dwa podciągi zbieżne do różnych granic.
\end{enumerate}
\end{tw}

\begin{tw}
(Warunek Cauchy'ego)

$$ (\forall \epsilon > 0)(\exists k \in \mathbb{N})(\forall m,n \geq k)
   (|a_m - a_n| < \epsilon)
$$
\end{tw}

\begin{fakt}
\

\begin{enumerate}
\item Każdy ciąg zbieżny spełnia warunek Cauchy'ego.
\item Każdy ciąg spełniający warunek Cauchy'ego jest ograniczony.
\item Każdy ciąg spełniający warunek Cauchy'ego zawiera podciąg zbieżny.
\end{enumerate}
\end{fakt}

\begin{tw}
(Cauchy)

Ciąg liczb rzeczywistych jest zbieżny $\iff$ spełnia warunek Cauchy'ego.
\end{tw}

\section{Szeregi liczbowe}

\begin{de}
Niech $(a_n)$ będzie ciągiem liczb rzeczywistych. Szeregiem o wyrazach 
$a_n$,\\ $n=1,2,...$, nazywamy \textbf{ciąg sum częściowych}:
$$
  S_n = a_1 +...+ a_n = \sum_{k=1}^{n} a_k
$$
\end{de}

\begin{de}
\

Granicą ciągu sum częściowych, o ile istnieje, nazywa się sumę szeregu
i jest oznaczana $ \sum_{k=1}^{\infty} a_k = a_1 + a_2 + ... $

Mówimy wtedy że szereg jest zbieżny. W przeciwnym razie - rozbieżny.
\end{de}

\begin{de}
(n-ta reszta)

$ r_n = \sum_{k=n+1}^{\infty} a_k = a_{n+1} + a_{n+2} + ... $
to \textbf{n-ta reszta} szeregu $ \sum_{k=1}^{\infty} $
\end{de}

\subsection{Podstawowe własności szeregów}

\begin{enumerate}
\item Jeśli $ \sum_{k=1}^{\infty}a_k $ i $ \sum_{k=1}^{\infty}b_k $
      są zbieżne ,to $ \sum_{k=1}^{\infty}a_k+b_k $ jest zbieżny
      oraz $ \sum_{k=1}^{\infty}a_k+b_k =  \sum_{k=1}^{\infty}a_k +
      \sum_{k=1}^{\infty}b_k$
\item Jeśli $ \sum_{k=1}^{\infty}a_k $ jest zbieżny to:
      $ \sum_{k=1}^{\infty}c \cdot a_k = 
      c \sum_{k=1}^{\infty}a_k $
\item Jeżeli $ \sum_{k=1}^{\infty}a_k $ jest zbieżny to ciąg reszt
      $ r_n = \sum_{k=n+1}^{\infty}a_k $ jest zbieżny do 0.
\item (Warunek Cauchy'ego)

      $$
      \sum_{k=1}^{\infty}a_k \text{ jest zbieżny}
      \iff
      (\forall \epsilon>0)(\exists k \in \mathbb{N})(\forall m \in \mathbb{N})
      (\underbrace{|a_{k+1}+ ... + a_{k+m}|}
      _{\substack{\text{m-ta suma częściowa}\\ \text{k-tej reszty}}}
      < \epsilon)
      $$
\end{enumerate}

\subsection{Kryteria zbieżności szeregów}

\begin{enumerate}
\item Leibniz

     Jeżeli ciąg $(a_n)$ jest malejący i zbieżny
     do zera, to szereg $ \sum_{n=1}^{\infty} (-1)^n a_n $
     jest zbieżny.
\item Porównawcze

      Jeśli $ 0 \leq b_n \leq a_n $ (ciągi nieujemne) i szereg
      $ \sum_{n=1}^{\infty}a_n $ jest \textbf{zbieżny} to
      $ \sum_{n=1}^{\infty}b_n $ jest \textbf{zbieżny} i
      $ \sum_{n=1}^{\infty}b_n \leq \sum_{n=1}^{\infty}a_n $

      Jeśli $ 0 \leq a_n \leq b_n $ (ciągi nieujemne) i szereg
      $ \sum_{n=1}^{\infty}a_n $ jest \textbf{rozbieżny} to
      $ \sum_{n=1}^{\infty}b_n $ jest \textbf{rozbieżny}
\item d'Alambert

      $ \sum_{n=1}^{\infty}a_n $, $a_n >0$\\
      \\
      $ \lim_{n \to \infty} \frac{a_{n+1}}{a_n} < 1 $ - szereg zbieżny\\
      $ \lim_{n \to \infty} \frac{a_{n+1}}{a_n} = 1 $ - nie można określić zbieżności\\
      $ \lim_{n \to \infty} \frac{a_{n+1}}{a_n} > 1 $ - szereg rozbieżny
\item Cauchy

      $ \sum_{n=1}^{\infty}a_n $, $a_n >0$\\
      \\
      $ \lim_{n \to \infty} \sqrt[n]{a_n} < 1 $ - szereg zbieżny\\
      $ \lim_{n \to \infty} \sqrt[n]{a_n} = 1 $ - nie można określić zbieżności\\
      $ \lim_{n \to \infty} \sqrt[n]{a_n} > 1 $ - szereg rozbieżny
\item całkowe

      Dla szeregu $ \sum_{n=1}^{\infty}a_n $ tworzymy funkcję $f(x)$ taką,
      że $f(n) = a_n $ dla każdego n.
      $f(x)$ - malejąca i dodatnia dla $x \geq n_0 $ dla pewnego $n_0$
      Szereg jes zbieżny wtedy i tylko wtedy gdy całka
      $$
      \int_{n_0}^{\infty} f(x) dx
      $$
      jest zbieżna.
\item warunek konieczny zbieżności szeregu

      $$
      \sum_{n=1}^{\infty}a_n  \text{ jest zbieżny}
      \implies
      \lim_{n \to \infty} a_n =0
      $$

      i równoważnie:

      $$
      \lim_{n \to \infty} a_n \neq 0
      \implies
      \sum_{n=1}^{\infty}a_n  \text{ jest rozbieżny}
      $$
      
\end{enumerate}

\begin{center}
\textbf{OSZACOWANIA DLA KRYTERIUM PORÓWNAWCZEGO}
\end{center}

$$
\sum_{n=1}^{\infty} \frac{1}{n^{\alpha}}
\begin{cases}
\alpha > 1 -\text{ zbieżny}
\\
\alpha \leq 1 -\text{ rozbieżny}
\end{cases}
$$
$$
\sin x < x
$$
$$
\ln x < x-1
$$
$$
\sin x \geq \frac{2}{\pi}x, \hspace{1cm} x \in [0;\frac{\pi}{2}]
$$
$$
\tg x \leq \frac{4}{\pi}x, \hspace{1cm} x \in [0;\frac{\pi}{4}]
$$

\subsection{Szeregi bezwzględnie zbieżne}

\begin{de}
$ \sum_{n=1}^{\infty}a_n $ jest bezwzględnie zbieżny, jeśli
$ \sum_{n=1}^{\infty}|a_n| $ jest zbieżny.
\end{de}

\begin{tw}
Szereg bezwzględnie zbieżny jest zbieżny. Ponadto
$ |\sum_{n=1}^{\infty}a_n| \leq \sum_{n=1}^{\infty}|a_n| $
\end{tw}

\begin{tw}
Szereg $ \sum_{n=1}^{\infty}a_n $ jest warunkowo zbieżny, gdy jest zbieżny,
ale nie jest bezwzględnie zbieżny. (np. kryterium Leibniza)
\end{tw}

\begin{tw}
Szeregi bezwzględnie zbieżne są przemienne, to znaczy:
\begin{center}
$ \sum_{n=1}^{\infty}a_n $ - bezwzględnie zbieżny\\
$ m_1, m_2, ...$ permutacje zbioru $\mathbb{N}$
to:\\
$ \sum_{n=1}^{\infty}a_{m_n} $ jest zbieżny oraz
$ \sum_{n=1}^{\infty}a_{m_n} = \sum_{n=1}^{\infty}a_n $
\end{center}
\end{tw}

\begin{tw}
(Riemann)

Jeśli $ \sum_{n=1}^{\infty}a_n $ jest warunkowo zbieżny to
$ (\forall r \in \mathbb{R} \cup \{\pm \infty \}) $
istnieje taka permutacja $m_1, m_2, ... $ taka, że
$$ \sum_{n=1}^{\infty}a_{m_n} = r $$

Istnieje też permutacja $k_1, k_2, ... $ taka, że $ \sum_{n=1}^{\infty}a_{k_n} $
jest rozbieżny.

\end{tw}


\begin{tw}
(Mnożenie szeregów bezwzględnie zbieżnych)

Jeśli $ \sum_{n=1}^{\infty}a_n $, $ \sum_{n=1}^{\infty}b_n $
są bezwzględnie zbieżne to:

$$
(\sum_{n=1}^{\infty}a_n)  (\sum_{n=1}^{\infty}b_n) = \sum_{n=1}^{\infty}c_n 
$$
gdzie $ c_n = a_1b_n + a_2b_{n-1} + ... + a_nb_1 $
\end{tw}
 
\begin{tw}
Do obliczania zbieżności bezwzględnej szeregów można używać kryteriów
d'Alamberta i Cauchy'ego.
\begin{center}
$ \lim_{n \to \infty} |\frac{a_{n+1}}{a_n}| < 1 $ - szereg bezwzględnie zbieżny\\
$ \lim_{n \to \infty} \sqrt[n]{|a_n|} < 1 $ -       szereg bezwzględnie zbieżny\\
(analogicznie dla granicy > 1)
\end{center}
\end{tw}

\section{Szeregi potęgowe}

\begin{de}
Szereg
$$
\sum_{n=1}^{\infty}a_nx^n
$$
nazywamy szeregiem potęgowym.
\end{de}

\begin{tw}
Jeśli $ \sum_{n=1}^{\infty}a_nx^n $ jest zbieżny dla $x=x_0 \neq 0$, to jest zbieżny bezwzględnie
$ (\forall x \in (-|x_0|,|x_0|)) $
\end{tw}

\begin{de}
Promień zbieżności szeregu $ \sum_{n=1}^{\infty}a_nx^n $

$$ r = \sup\{|x_0|: \sum_{n=1}^{\infty}a_nx^n \text{ jest zbieżny }\} $$

\begin{itemize}
\item Jeśli $x \in (-r; r)$ to $ \sum_{n=1}^{\infty}a_nx^n $ jest zbieżny.
\item Jeśli $x \notin (-r; r)$ to $ \sum_{n=1}^{\infty}a_nx^n $ jest rozbieżny.
\item Jeśli $x = \pm r $ to $ \sum_{n=1}^{\infty}a_n $ moze być zbieżny lub rozbieżny.
\end{itemize}

\end{de}

\subsection{Wyznaczanie promienia zbieżności}

$$
r = \left[
    \frac{1}{
      \lim\limits_{n \to \infty} | \frac{a_{n+1}}{a_n} |
    }
    \right]
\quad
\text{lub}
\quad
r = \left[
    \frac{1}{
      \lim\limits_{n \to \infty} | \sqrt[n]{a_n} |
    }
    \right]
$$

Następnie należy sprawdzić zbieżność dla $ x= \pm r $

\section{Funkcje}

\subsection{Granice funkcji}

\begin{de}
(Punkt skupienia zbioru)

$ \emptyset \neq D \subseteq \mathbb{R} $

$a \in \mathbb{R}$ jest lewostronnym punktem skupienia zbioru D, gdy
istnieje ciąg $(d_n)$ taki, że $ d_n \in D $, $ d_n > a $ i $ \lim d_n = a $

$a \in \mathbb{R}$ jest prawostronnym punktem skupienia zbioru A, gdy
istnieje ciąg $(d_n)$ taki, że $ d_n \in D $, $ d_n < a $ i $ \lim d_n = a $

$a$ jest punktem skupienia zbioru $D$ gdy jest lewo- lub prawostronnym punktem skupienia.

Intuicyjnie $a$ jest punktem skupienia zbioru $D$ gdy dowolnie blisko
$a$ znajduje się nieskończenie wiele liczb ze zbioru $D$.
\end{de}

\underline{PRZYKŁADY:}

\begin{itemize}
\item $D$ - zbiór skończony - brak punktu skupienia
\item $\mathbb{N}$ - brak punktu skupienia
\item $ \{ 1, \frac{1}{2}, \frac{1}{3}, ...\} $ - punkt skupienia $a=0$
\item $[0,1]$ - każdy punkt jest punktem skupienia
\item $\mathbb{Q}$ - każda liczba rzeczywista jest punktem skupienia
\end{itemize}

\begin{de}
(granica funkcji)

Niech: $f: D \to \mathbb{R} $. Niech $a$ będzie lewostronnym punktem skupienia zbioru $D$.
$ \lim_{x \to a^-} f(x) = g $, gdy dla każdego ciągu $ x_n \to a, x_n \in D, x_n<a $
zachodzi $ \lim_{x \to a^-} f(x_n) = g $. Liczbę $g$ nazywamy granicą lewostronną funkcji $f$
w punkcie $a$.

Analogicznie $ \lim_{x \to a^+} f(x) = g $ (punkt skupienia prawostronny i $x_n>a$) nazywamy 
granicą prawostronną.
\end{de}

\begin{de}
$$
\lim_{x \to \pm \infty} f(x) = g \text{, gdy } (\forall x_n \in D), \lim x_n = \pm \infty
\text{, mamy } \lim f(x_n) = g
$$
\end{de}

\textbf{UWAGA}
Arytmetyka granic jest analogiczna jak w przypadku ciągów. Dotyczy to również twierdzenia 
o trzech ciągach. (w tym wypadku - trzech funkcjach)

\begin{tw}
(granica funkcji żłożonej)

$$
\text{Jeśli}
\lim_{x \to a} f(x) = A, 
\lim_{y \to A} g(y) = B\text{, to}
\lim_{x \to a} g(f(x)) = B
$$
\end{tw}

\subsection{Ciągłość funkcji}

\begin{de}
(Heine)

Niech $ f: X \to Y; X, Y \subseteq \mathbb{R} $. Funkcja $f$ jest ciągłą w punkcie a jeśli:

$$
(\forall x_n \in X)((x_n \to a) \implies (f(x_n) \to f(a)))
$$

\begin{itemize}
\item Funkcja jest ciągła lewostronnie, gdy:
$$
(\forall x_n \in X)((x_n \to a^-) \implies (f(x_n) \to f(a)))
$$
\item Funkcja jest ciągła prawostronnie, gdy:
$$
(\forall x_n \in X)((x_n \to a^+) \implies (f(x_n) \to f(a)))
$$
\end{itemize}

Fukcja jest ciągła w $a \iff f$ jest ciągła lewo- i prawostronnie w a.
\end{de}

\begin{tw}
\

Suma, różnica, iloczyn, iloraz funkcji ciągłych jest ciągły.
Złożenie dwóch funkcji ciągłych jest funkcją ciągłą.
\end{tw}

\begin{de}
(Cauchy)

Funkcja $ f: D \to \mathbb{R} $ jest ciągła w punkcie $ x_0 $ gdy:

$$
(\forall \epsilon > 0)(\exists \delta)(\forall x)((|x-x_0|<\delta) \implies (|f(x)-f(x_0)|<\epsilon))
$$

Powyższy warunek nazywamy warunkiem Cauchy'ego ciągłości funkcji w punkcie $x_0$.
\end{de}

\begin{de}
(jednostajna ciągłość)

Funkcja $ f: D \to \mathbb{R} $ jest jednostajnie ciągła, gdy 
$$
(\forall \epsilon > 0)(\exists \delta > 0)(\forall x,x')((|x-x'|<\delta) \implies
(|f(x)-f(x')| < \epsilon))
$$

Funkcja jednostajnie ciągła jest ciągła.
\end{de}

\begin{de}
(warunek Lipschitza)

Funkcja $ f:\mathbb{R} \to \mathbb{R} $ spełnia warunek Lipschitza ze stałą $L$,
gdy dla dowolnych $x_1, x_2 \in \mathbb{R}$ zachodzi nierówność
$$
|f(x_1)-f(x_2)| \leq L|x_1-x_2|
$$
Każda funkcja spełniająca warunek Lipschitza jest jednostajnie ciągła.
\end{de}

\begin{tw}
\

Niech $f:(a,b) \to \mathbb{R}$ będzie funkcją różniczkowalną. Wówczas $f$
spełnia warunek Lipschitza ze stałą $L$ wtedy i tylko wtedy gdy jej pochodna
jest ograniczona (z góry lub z dołu) przez $L$.
\end{tw}

\begin{de}
Zbiór $Z$ jest wypukły gdy $a,b \in Z \implies [a,b] \in Z$
\end{de}

\begin{tw}
(Darboux - o przyjmowaniu wartości pośrednich)

Jeśli $f: Z \to \mathbb{R}$ jest ciągła i $Z$ jest wypukły w $\mathbb{R}$, to $f[Z]$ jest wypukły.

\underline{Wniosek:}

Jeśli $f$ jest ciągła na zbiorze wypukłym $Z$ i $ f(x)<t<f(y) $, to $ t \in f[Z] $, 
funkcja przyjmuje wartości pośrednie $f(x)$ i $f(y)$
\end{tw}

\begin{tw}
Jeśli $ f:[a,b] \to \mathbb{R} : f $ jest ciągła, to $ [a, b] $
jest przedziałem domkniętym. W szczegolności $f$ przyjmuje wartość 
największą i najmniejszą.
\end{tw}

\begin{tw}
(O funkcji odwrotnej)

Jeśli $ f: [a,b] \to f([a,b]) \subseteq \mathbb{R} $ jest ciągła i różnowartościowa
to funkcja odwrotna $ g=f^{-1}: f([a, b]) \to [a,b] $ jest ciągła.
\end{tw}

\subsection{Pochodna funkcji}

\begin{de}
(iloraz różnicowy w punkcie $ a \in D_f $)

$$
\frac{f(x) - f(a)}{x - a}
$$
alternatywnie dla $ x-a = h $
$$
\frac{f(a+h) - f(a)}{h}
$$

Pochodną funkcji nazywamy granicę ilorazu róznicowego gdy $h \to 0$
\end{de}

\begin{tw}
$$
f \text{ jest różniczkowalna w }a \implies f \text{ jest ciągła w } a
$$
\end{tw}
\subsection{Ekstrema funkcji}

\begin{de}
\

$f: D \to \mathbb{R}$ ma maksimum lokalne w punkcie $a \in D$, gdy dla 
pewnego przedziału $ (a - \delta, a+\delta),(\delta>0) $ (otoczenie punktu $a$)
i dla wszystkich $ x \in (a - \delta, a+\delta) \cap D $ mamy
$ f(x) \leq f(a) $
\end{de}

\begin{tw}
Jeśli funkcja jest różniczkowalna w punkcie $a$ i ma w $a$ ekstremum lokalne,
to $f'(a)=0$. (warunek konieczny istnienia ekstremum)
\end{tw}

\begin{tw}
(Rolle'a)

Jeśłi $f:[a,b] \to \mathbb{R}$ jest ciągła i różniczkowalna w $(a,b)$ oraz
$f(a) = f(b)$, to istnieje $c\in(a,b)$ takie że $f'(c)=0$.
\end{tw}

\begin{tw}
(Lagrange'a)

Jeśli $f:[a,b]\to\mathbb{R}$ jest ciągła i różniczkowalna w $(a,b)$, to 
istnieje $c \in (a,b)$, takie że
$$
f'(c) = \frac{f(b) - f(a)}{b-a}
$$
\end{tw}

\begin{tw}
(Cauchy'ego)

Jeśli $f:[a,b] \to \mathbb{R}$ i $g:[a,b] \to \mathbb{R}$
są ciągłe i różniczkowalne w $(a,b)$ i 
$ (\forall x \in (a,b))(g'(x) \neq 0) $, to istnieje $c \in (a,b)$ takie, że:
$$
\frac{f(b) - f(a)}{g(b)-g(a)} = \frac{f'(c)}{g'(c)}
$$
\end{tw}
\subsection{Asymptotyczne tempo wzrostu funkcji}

\begin{de}
(Notacja "O")
$$
f(x) = O(g(x)) \iff \lim_{x \to \infty} \left|\frac{f(x)}{g(x)}\right|< \infty
$$
\end{de}

\begin{de}
(Notacja "o")
$$
f(x) = o(g(x)) \iff \lim_{x \to \infty} \frac{f(x)}{g(x)}=0
$$
\end{de}

\begin{de}
(Notacja "$\Theta$")
$$
f(x) = \Theta(g(x)) \iff 0<\lim_{x \to \infty}
\left|\frac{f(x)}{g(x)}\right|< \infty
$$
\end{de}

\subsection{Reguła de l'Hospitala}

Jeśli $\lim\limits_{x \to a} f(x) = \lim\limits_{x \to a} g(x) \in \{0, \pm\infty \}$
oraz istnieje granica $\lim\limits_{x \to a} \frac{f'(x)}{g'(x)}$
to

$$
\lim_{x \to a} \frac{f(x)}{g(x)} = \lim_{x \to a} \frac{f'(x)}{g'(x)}
$$
\subsection{Wzór Taylora i Maclaurina}

Jeśli funkcja jest n-krotnie różniczkowalna w $[a,b]$, to
$$
f(b) = f(a) + \frac{b-a}{1!} \cdot f'(a) + ... +
       \frac{(b-a)^2}{2!} \cdot f''(a) + ... +
       \frac{(b-a)^{n-1}}{(n-1)!} \cdot f^{(n-1)}(a) + R_n
$$
gdzie
$$
R_n = \frac{(x-x_0)^n}{n!}f^{(n)}(c_n), \ \ a<c_n<b
$$

Wzór Taylora dla $a=0$ nazywamy wzorem Maclaurina.

\begin{tw}
Jeżeli ciąg reszt $R_n$ w rozwinięciu Taylora (w szczególności Maclaurina)
funkcji $f$ jest zbieżny do $0$ dla każdego $x$ z pewnego otoczenia $U$
punktu $x_0$, szereg
$$ f(x) = \sum_{n=0}^{\infty} \frac{(x-x_0)^n}{n!} \cdot f^{(n)}(x_0) $$
jest zbieżny dla każdego $x \in U$

Taki szereg nazywamy szeregiem Taylora (Maclaurina) funkcji f.
\end{tw}

\subsection{Różniczkowanie szeregów potęgowych}

\begin{tw}
Szeregi $ \sum_{n=1}^{\infty}c_n x^n $ i $ \sum_{n=1}^{\infty}n c_n x^{n-1} $
mają takie same promienie zbieżności i zachodzi wzór
$$ \left(\sum_{n=1}^{\infty}c_n x^n \right)' =
\sum_{n=1}^{\infty}n c_n x^{n-1} $$

dla $ |x| < R $ - promień zbieżności
(różniczkowanie wyraz po wyrazie)
\end{tw}

\section{Całki}

\subsection{Metody liczenia całek nieoznaczonych}

\subsubsection{Całkowanie przez części}

Jeżeli $f$ i $g$ mają ciągłe pierwsze pochodne to:

$$
\int f(x) \cdot g'(x) dx = f(x) \cdot g(x) - \int f'(x) \cdot g(x) dx
$$

\subsubsection{Całkowanie przez podstawienie}

Jeżeli funkcję $f(x)$ można zapisać w postaci
$$
f(x) = g(h(x)) \cdot h'(x)
$$
gdzie funkcja $h(x)$ ma ciągłą pochodną, to
$$
\int f(x) dx = \int g(t) dt
$$
gdzie $ t = h(x) $

\subsection{Całki oznaczone}
Jeśli
$$
\int f(x) dx = F(x)
$$
to
$$
\int_a^b f(x) dx = F(b) - F(a)
$$

\begin{tw}
(Podstawowe twierdzenie rachunku całkowego)

Jeśli funkcja $f: \mathbb{R} \to \mathbb{R}$ jest ciągła, to
$$
\frac{d}{dx} \left(
                    \int_a^x f(t) dt 
             \right)
             = f(x)
$$
Wynika z tego iż całkowanie jest operacją odwrotną do różniczkowania.
\end{tw}

\subsection{Całki niewłaściwe}

\subsubsection{Całki niewłaściwe pierwszego rodzaju}

Są to całki określone na nieograniczonym przedziale całkowania
$[a, \infty),(-\infty, b]$ lub $(-\infty,\infty)$. Liczy je się 
za pomocą granic.

$$
\int_a^{\infty} f(x) dx = \lim_{T \to \infty} \int_a^T f(x) dx
$$
$$
\int_{-\infty}^b f(x) dx = \lim_{T \to -\infty} \int_T^b f(x) dx
$$
$$
\int_{-\infty}^{\infty} f(x) dx =
\int_{-\infty}^a f(x) dx + \int_a^{\infty} f(x) dx 
$$

Całka niewłaściwa jest
\begin{itemize}
\item zbieżna (granica właściwa równa dowolnej liczbie) 
\item rozbieżna do $\pm \infty$ (granica niewłaściwa)
\item rozbieżna (granicy nie da się określić)
\end{itemize}

\subsection{Metody aproksymacji (przybliżania) całek oznaczonych}

\subsubsection{Przybliżanie całki Riemanna przez sumy całkowe}

\subsubsection{Metoda trapezów}

$P=$ podział $[a,b]$ na $n$ równych części

\begin{figure}[h]
\centering
\includegraphics[scale=0.6]{trapez.png}

pole trapezu $ \frac{f(x_{i-1}) + f(x_i)}{2} \Delta x_i ,
\ \ \Delta x_i=\frac{b-a}{n}$
\end{figure}

$$ 
\int_a^b f(x) dx \approx \frac{b-a}{2n} \sum_{i=1}^n \big(f(x_{i-1})+f(x_i)\big) = 
\frac{b-a}{2n} \big(f(a) +2f(x_1) + ... + 2f(x_{n-1}) + f(b) \big)
$$

\subsubsection{Metoda Simpsona}

$P=$ podział $[a,b]$ na $n$ równych części, $n$ - parzyste,
$\Delta x_i=\frac{b-a}{n} = h$

$$
\int_a^b f(x) dx \approx
\frac{b-a}{3n}
\bigg[
f(a) +4f(x_1) + 2f(x_2) + 4f(x_3) + 2f(x_4) +...+
4f(x_{n-1}) + f(b)
\bigg]
$$

$n$ - ilość punktów podziału

błąd $E_n$ można przybliżyć jako

$$
0 \leq E_n \leq \frac{K}{180n^4}(b-a)^5
\text{, gdzie }
K = \max\{|f^{(4)}(x)|: x \in [a,b]\}
$$
\newpage
\underline{Przykład zastosowania}

\begin{figure}[h!]
\centering
\includegraphics[scale=0.4]{plama.png}

plama ropy
\end{figure}

Według wzoru
$$
P = \int_0^{10} [g(x)-h(x)] dx \approx
\frac{10}{3\cdot10} \bigg[0.5 + 4\cdot1.1 + 2\cdot1.3 + 2\cdot1.4 + ... + 0.2 \bigg]
$$

\subsubsection{Wielomian interpolacyjny Lagrange'a}

Danych jest $(n+1)$ punktów $x_0,...,x_n \in \mathbb{R}$, w których pewna 
funkcja przyjmuje wartości (pomiary) $y_0,...,y_n \in \mathbb{R}$
Zakładamy że funkcja ta jest ciągła (opisuje zjawisko fizyczne ciągłe)

Wtedy istnieje (dokładnie jeden) wielomian stopnia $n$ przyjmujący wartości 
$y_0,...,y_n$ w punktach $x_0,...,x_n$.

$$
W(x) = y_0u_0(x) + y_1u_1(x) +...+ y_nu_n(x)
$$
gdzie
$$
u_k(x) = \frac
{(x-x_0)\cdot...\cdot(x-x_{k-1})(x-x_{k+1})\cdot...\cdot(x-x_n)}
{(x_k-x_0)\cdot...\cdot(x_k-x_{k-1})(x_k-x_{k+1})\cdot...\cdot(x_k-x_n)}
$$

Jeżeli $u_k$ dla pewnego $x$ przyjmuje wartość $1$ to dla innego $x$
przymuje wartość $0$.

\section{Funkcje wektorowe}

\begin{de}
Przestrzenie euklidesowe $\mathbb{R}^n$

$\mathbb{R}^n$ jest przestrzenią liniową (wektorową) z działaniami:
\begin{itemize}
\item dodawanie punktów (wektorów):
      $$
      (x_1,...,x_n) \pm (y_1,...,y_n) = (x_1+y_1,...,x_n+y_n)
      $$
\item mnożenie przez skalar:
      $$
      \alpha (x_1,...,x_n) = (\alpha x_1,...,\alpha x_n)
      $$
\end{itemize}

$\overline{0} = \vec{0} = (0,...,0)$ nazywamy wektorem zerowym
\end{de}

\begin{de}
Wektor $\overline{AB}$ $a,b \in \mathbb{R}^n$, o początku w punkcie $A$ i końcu
w $B$
$$
\overline{AB} = (b_1-a_1,...b_n-a_n)
$$
gdzie $a_n$ to współrzędne punktu $A$, a $b_n$ to współrzędne punktu $B$.
\end{de}

\subsection{Rachunek wektorowy}

\begin{de}
(dodawanie wektorów)
$$
\overline{AB}+\overline{BC}=\overline{AC}
$$
\end{de}

\begin{de}
(iloczyn skalarny)
$$
\overline{x} \cdot \overline{y} = x_1y_1+...+x_ny_n
\text{, gdzie }
\overline{x} = (x_1,...,x_n) \ 
\overline{y} = (y_1,...,y_n)
$$
\end{de}

\begin{tw}
(właśności iloczynu skalarnego)

\begin{enumerate}
\item $ \overline{x} \cdot \overline{y} = \overline{y}  \cdot \overline{x} $
\item $ \overline{x}\cdot (\overline{y}+\overline{z}) =
        \overline{x}\cdot\overline{y}+\overline{x}\cdot\overline{z} $
\item $ (\alpha\overline{x}) \cdot \overline{y}=
        \alpha(\overline{x} \cdot \overline{y})= 
        \overline{x} \cdot (\alpha\overline{y}) $
\end{enumerate}
\end{tw}

\begin{de}
(iloczyn wektorowy w $\mathbb{R}^3$)
$$
\overline{x} \times \overline{y} =
 \overline{i} (x_2y_3-x_3y_2)
+\overline{j} (x_3y_1-x_1y_3)
+\overline{k} (x_1y_2-x_2y_1) = \\
((x_2y_3-x_3y_2), (x_3y_1-x_1y_3), (x_1y_2-x_2y_1))
$$
\end{de}

\begin{tw}
(właśności iloczynu wektorowego)

\begin{enumerate}
\item $\overline{a} \times (\overline{b} \times \overline{c})
      = (\overline{a} \cdot \overline{c}) \times \overline{b}
      - (\overline{a} \cdot \overline{b}) \times \overline{c} $
\item $(\overline{a} \times \overline{b}) \cdot \overline{c} = 
        \overline{a}\cdot(\overline{b}\times\overline{c}) $
\item $ \overline{a}\times\overline{b}=
       -\overline{b}\times\overline{a} $
\end{enumerate}
\end{tw}

\begin{de}
(norma euklidesowa wektora)

$$
||\overline{x}|| = \sqrt{x_1^2+...+x_n^2}
$$
$$
||x||^2 = \overline{x} \cdot \overline{x} = \overline{x}^2
$$
\end{de}

\begin{de}
(Nierówność Cauchy'ego-Schwarza)
$$
\overline{x} \cdot \overline{y} \leq 
||\overline{x}|| \cdot ||\overline{y}||
$$
\end{de}

\begin{de}
(kula w $\mathbb{R}^n$)
$$
K(\overline{a}, r) =
\{\overline{x} \in \mathbb{R}^n:||\overline{x}-\overline{a}|| < r\}
$$
gdzie $\overline{a}$ to środek kuli a $r$ to dodatni promień kuli.
\end{de}

\subsection{Ciagi i funkcje wektorowe}


\begin{de}
(ciąg w $\mathbb{R}^n$)
$$
f: \mathbb{N} \to \mathbb{R}^n
\text{, co oznaczamy }
(\overline{x_n})_{n=1}^{\infty}
$$
Intuicyjnie, jest to ciąg wektorów.
\end{de}

\begin{de}
(granica ciągu w $\mathbb{R}^n$)
$$
lim_{n \to \infty} = \overline{x_n} = \overline{x} \in \mathbb{R}^n
\text{, gdy }
(\forall \epsilon>0)(\exists k \in \mathbb{N})(\forall n\geq k)
(||\overline{x_n}-\overline{x}||<\epsilon)
$$
\end{de}

\begin{tw}
\

Jeśli $\overline{x}=(x_1,...,x_n), \overline{x_n} = (x_{n1},...,x_{nm})$
to
$$\lim \overline{x_n} = \overline{x} \iff
(\forall 1 \leq i \leq m)(\lim\limits_{n \to \infty} x_{ni} = x_i)$$
\end{tw}

\begin{de}
(Granica funkcji wektorowej)

$ f: D \to \mathbb{R}^n : D \subseteq \mathbb{R}$
$t_0$ - punkt skupienia $D$

$$
\lim_{t\to t_0} f(t) = \overline{x}
\text{, gdy }
(\forall \epsilon>0)(\exists \delta>0)(|t-t_0|<\delta \implies
||f(t)-\overline{x}||<\epsilon)
$$
(Granica funkcji wektorowej jest granicą funkcji składowych)
\end{de}

\begin{tw}
Funkcja wektorowa jest ciągła gdy wszystkie jej składowe są ciągłe.
\end{tw}

\begin{tw}
Jeżeli dwie funkcje wektorowe są ciągłe to ich suma, różnica, iloczyn wektorowy
i skalarny funkcji wektorowych są ciągłe
\end{tw}

\begin{tw}
Pochodna funkcji wektorowej to funkcja wektorowa o składowych będących
pochodnymi skłądowych różniczkowanej funkcji. Zatem różniczkujemy
"po składowych"
\end{tw}

\begin{tw}
Całka nieoznaczona $ \int f(x)dx = F(x) + \overline{W}$z funkcji wektorowej
to funkcja pierwotna wyznaczana z dokładnością do stałego wektora czyli taka,
że $F'(x) = f(x)$

Całka nieoznaczona z funkcji wektorowej to funkcja o składowych równych całkom
nieoznaczonym całkowanej funkcji.
\end{tw}

\begin{tw}
Całka oznaczona z funkcji wektorowej to wektor o składowych równych
całkom oznaczonym z funkcji składowych całkowanej funkcji.
\end{tw}

\subsection{Interpretacja fizyczna}

Jeśli $\overline{r}(t) = (x(t), y(t), z(t))$ opisuje położenie punktu, to
$\overline{r}'(t) $ to wektor prędkości w chwili $t$.

Analogicznie $\overline{r}''(t)$ to wektor przyspieszenia.



\newpage
\includepdf{wzory.pdf}
\end{document}
